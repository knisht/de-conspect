\documentclass[12pt]{article}
\usepackage[T2A]{fontenc}
\usepackage[utf8]{inputenc}
\usepackage[russian]{babel}
\newcommand*\diff{\mathop{}\!\mathrm{d}}
\usepackage{mathtools}
% % \documentclass[20]{extreport}
\usepackage[%
    left=0.75in,%
    right=0.75in,%
    top=1.0in,%
    bottom=1.0in,%
    paperheight=11in,%
    paperwidth=8.5in%
]{geometry}%

\renewcommand{\baselinestretch}{1.5} 

\title{Diffoory cheatsheet}
%\author{knisht}
\date{September 2018}

\begin{document}

\maketitle

\section{Теоретическое говно, которое надо помнить}
Короче, есть два вида уравнений, в канонической и симметричной форме.\\
Каноническая $y' = f(x, y)$ \\
Симметричная $M(x,y)dx + N(x,y)dy = 0$
\subsection{Каноническая}
Каноническое уравнение задается на области и части границы этой области. \\
Что надо знать нам: У любой области есть граница, но не на всей границе задается наше уравнение. \\
\textit{Граница нужна для решения задачи Коши, поэтому советую обратить внимание на говно, которое здесь написано.} \\
Короче область существования задается в основном ОДЗ.(лорифмы, отрицательная степень у икса и прочее говно) в основном это "плохая граница"(на которй наше уравнение отсасывает), кроме того случая когда оно продолжимо на границу (как $xlnx$). А "хорошая граница" это такое условие, в котором дифур то не существует совсем. Например в $M(x,y)y' \\\ M(x,y)$ - граница.

\subsection{Симметричная}
А вот тут говнище полное, кроме области и ее границы, существует говно - особые точки. В них вообще ну полный пиздец и непредсказуемость. Особые точки, это случай, когда $M^2 + N^2 = 0$ тут дифуры тоже нет никакой, вырождается все, от того и непредсказуемо. Это тоже невероятно важно для решения вашей горячо любимой задачи Коши. \\
Уравнение в этой форме сводится к канону, но про особые точки надо помнить.

\section{С чего начинается решение дифуры?}
В первую очередь - с ОДЗ. Потом, если уравнение в симметричной - особые точки надо найти(из условия выше)

\section{Задача Коши для самых маленьких}
Да, вы не ослышались.Я попробую максимально коротко. \\
Короче говоря, вот вы нашли всякие решения, одно из них общее, в каком оно может быть виде? \\
$ y = f(x, C)$ - best of the best, то что надо \\
$ x = f(y, C)$ - неплохо \\
$ C = f(x, y)$ - общий интеграл называется, уже хуже \\
$ 0 = f(x, y, C)$ - ебать ты сосешь, чувак(иха) \\
В чем состоит решение залачи Коши? В том что вам надо найти конкретную функцию, то бишь избавиться от неизвестного $С$ ну и конечно же найти максимальный интервал существования. Первое это в общем то легко, подставляете, выводите С. Теперь, когда мы избавились от одной неизвестной, надо попробовать выразить все это дело в первый или второй вид. Попутные действия будут добавлять нам всякие точки разрыва. Ну и отвечая на вопрос, что же в итоге является максимальном интервалом существования - это интервал, внутри которого вашего точка. Слева и справа либо бесконечность, либо что то из следующего: ОДЗ, граница, гладкое пересечение со специальными решениями(которые из точек неединственности), а также особые точки, если есть.\\
Другое дело, если у вас нихрена не выражается - это уже так сказать "искусство", думать надо, график строить или другие методы. А так же, если вы $x = f(y)$ сделали, то надо интервал для $y$ делать. 
Ну вот и все. Искать все это - говно, говнище, говнилова. Но в этом весь "прикол".


\section{Типы уравнений}
\subsection{Важно помнить, что...}
\begin{enumerate}
\item $y' = \frac{\diff y}{\diff x} \Rightarrow $ при $y'$ в уравнении $dx \neq 0 \Rightarrow dx \not\equiv C$.
\item Чтобы делать замены, надо привести уравнение к его образцу (в левой части должен находиться $y'$ без коэффициентов) 
\item В уравнениях с разделёнными переменными $\diff x \not\equiv C$.
\item Интегралы неопределённые.
\item Частные решения состоят целиком из точек единственности.
\item Специальные решения состоят целиком из точек неединственности.
\item Вносить и выносить переменные под дробную степень нужно следя за знаком.
\item При решении уравнения Бернулли надо учитывать область определения функции (например, корни не могут быть отрицательными).
\item При решении дробно-линейного уравнения надо помнить, что общие точки двух линейных функций из условия пересекаются в точке неединственности.
\item Вас отчислят.
\end{enumerate}


\subsection{Уравнение с разделёнными переменными}
$\frac{y'}{f(y)} = g(x) \\$
ОДЗ: $f(y) \neq 0; x \not\equiv C\\$
\subsection{Уравнение с разделяющимися переменными}


\begin{enumerate}
\item $y' = f(x)g(y) \\ $
Помним про случай $g(y) = 0 .\\$ 
\item $f_1(x)g_1(y) \diff x + f_2(x) g_2(y) \diff y = 0 \\$
Рассматриваем: $f_2(x) = 0;g_1(y) = 0$ - эти решения всегда будут и решением исходного уравнения\\
Привод к канону - деление на $f_2(x)dx$ с проверкой
\end{enumerate}


\subsection{Линейные неоднородные уравнения}
$y' + a(x)y = b(x)$ \\
$y_{\text{общ}} = y_{\text{оо}} + y_{\text{чн}}$ \\
Решение методом вариации произвольной постоянной.
\begin{enumerate}
    \item ЛОУ - Решить $y' + a(x)y = 0$
    $y = C(x)f(x)\\$
    $y' = C'(x)f(x) + C(x)f'(x)$
    \item Ищем частные \\
    Подставить в исходное неоднородное уравнение, должно сократиться $C(x)$ \\
    Ну а теперь решаем новый дифур относительно $C'(x)$, подставляем обратно в $y$
\end{enumerate}


\subsection{Уравнение Бернулли}
$y' + a(x)y = b(x)y^n \And n \neq 0, 1\\$
$y'y^{-n} + a(x)y^{1-n} = b(x)\\$
Решается через замену $z = y^{1-n};  z' = (1-n)y^{-n}y'\\$
$z' + a(x)(1-n)z = b(x)(1-n)\\$
Это линейное неоднородное уравнение.


\subsection{Уравнение Риккати}
$y' + a(x)y + b(x)y^2 = c(x)\\$
Проанализировать степени слагаемых, найти частное решение $y_1.\\$
Сделать замену $y = y_1 + u\\$
$y'_1 + u' + a(x)(y_1+u) + b(x)(y1+u)^2 = \\
(y'_1 + a(x)y+b(x)y^2) + u' + a(x)u + 2b(x)y_1 u + b(x)u^2 = c(x)\\
u' + a(x)u + 2b(x)y_1u + b(x)u^2 = 0\\
u' + (a(x) + 2b(x)y_1(x))u = -b(x)u^2\\$
Это уравнение Бернулли при $n=2.$


\subsection{Однородные уравнения}
Однородным называется уравнение $f(x,y)$ , если $\forall k>0 : f(kx, ky) = k^nf(x, y) \\$
$y' = f(\frac{x}{y}) \\$
Решается через замену $y = ux; u = yx^{-1}; y' = u'x + u.\\$
После корректной замены должен исчезнуть $x$.$\\$
Получится уравнение с разделяющимися переменными.

\subsection{Дробно-линейные уравнения}
$y' = f({\frac{a_1 \cdot x + b_1 \cdot y + c_1}{a_2 \cdot x + b_2 \cdot y + c_2}})
\\ \\$
Решается через нахождение точки пересечения графиков $a_1 x + b_1 y + c_1$ и $a_2  x + b_2 y + c_2$ :

$\begin{cases}
a_1  x + b_1  y = - c_1 \\
a_2  x + b_2  y = - c_2
\end{cases}
\\$

Пусть точки пересечения это $x^*$ и $y^*$. $\\$
Тогда далее делается замена 
$\begin{cases} x = u + x^* & \diff x = \diff u \\ y = v + y^* & \diff y = \diff v \end{cases} \\$
Уравнение сводится к однородному.

\subsection{Обобщённо-однородные уравнения}
Однородные уравнения с разным весом переменных.
Есть два стула:
\begin{enumerate}
    \item $x \sim 1, y \sim m \\
    f(kx, k^m x) = k^n f(x, y) $ \\ 
    Решается заменой вида $y = z^m $  \\
    (не забыть про $y = -z^m$).
    \item $x \sim \alpha, y \sim \beta \\
    f(k^{\alpha}x, k^{\beta} y) = k^n f(x, y)\\$Подобрать $\alpha$ и $\beta$, ввести замену вида $y^{\alpha} = ux^{\beta}$
\end{enumerate}
Сводится к однородному уравнению. \\
Как понять когда юзать второй способ, а когда второй? Надо посмотреть на инвариантность x в исх уравнении. Если при $x=-x$(не забыв про $d(-x) = - dx$) уравнение остается неизменным - все чики пуки, изи первый способ. Иначе, стоит подумать о том, чтобы создать инвариантность введя вторую замену. Сразу использованием замены при однородности(это решит наши проблемы, не задумывайтесь об этом).
 
 
\subsection{Уравнения в полных дифференциалах}
$M(x, y) \diff x + N(x, y)\diff y = 0 \\$
Ввести функцию $U(x, y)$, где $U'_x \equiv M, U'_y \equiv N \\$
Убедиться, что $M'_y \equiv N'_x \\$
Решение: $\\$
$\displaystyle \int M \diff x = h(x,y) + C(y) = V \\ V'_y = N \\$
Найти $C_1$ как интеграл от $C'(y)\\$
Ответом будет выражение $h(x, y) = C_1$


\subsection{Уравнение с интегрирующим множителем}
$M(x, y) \diff x + N(x, y)\diff y = 0  \And M'_y - N'_x = F \neq 0\\$
Это уравнение - признак отчаяния, проверяется и используется в последнюю очередь.
Ввести интегрирующий множитель $\mu(x,y) \neq 0$, при котором $F = 0$, решать как уравнение в полных дифференциалах. $\\$
Вычисление $\mu:\\$
$(\mu M)'_y = (\mu N)'_x\\$
$\mu'_y M + \mu M'_y = \mu'_x N + \mu N'_x \\
\mu (M'_y - N'_x) = \mu'_x N - \mu'_y M\\
\mu = \frac{\mu'_x N - \mu'_y M}{F}$
Попытаться подобрать множитель через один из следующих способов: $\\$
\begin{enumerate}
\item $\mu(x,y) = \mu(x) \Rightarrow \mu'_y = 0 \\
\mu(x,y) = \mu(y) \Rightarrow \mu'_x = 0$
\item $\exists \omega(x, y) \neq 0 \\
\mu(x, y) = \mu(\omega(x, y))
\mu = \frac{\mu'_{\omega}(\omega'_x N - \omega'_y M)}{F}$
Это хитрое говно, надо выделять какие особенные функции и равнять их степени, а потом коэффициенты.
\end{enumerate}


\subsection{Уравнения, не разрешённые относительно производной}
$F(x, y, y') = 0\\$
1) Попробовать привести уравнение к виду $y' = f(x,y)$, это будет уравнение первого порядка $\\$
2) Попробовать привести уравнение к виду $y=f(x, y')$, ввести замену вида $p=\frac{\diff y}{\diff x} = y', \diff y = p \diff x\\$
После дифференцирования исходного уравнения выйдет: $p \diff x= f'_x(x,p)\diff x + f'_p(x, p) \diff p\\$
Привести уравнение к виду $\lambda(x, p) \cdot (M(x, p)\diff x + N(d,p)\diff p)=0$ , решить, рассматривая 2 случая, зануляя поочерёдно каждую скобку. $\\$
Возможны 2 варианта: $\\$ 
2.1) $x = x(p)$ , тогда $y = f(x(p), p)$, ответ будет в виде параметрической записи. $\\$
2.2) $p = p(x)$, тогда $y=f(x, p(x))$, это явная запись ответа.$\\$
2.3) Если не получается привести к двум верхним видам, то очень жаль. $\\$
3) $x = f(y, y') \\$
Ввести замену вида $p = y' = \frac{\diff y}{ \diff x}, \diff x = \frac{\diff y}{p}\\$
$\diff x = f'_y(y,p) \diff y + f'_p(y, p) \diff p\\$
$0 = (f'_y(y, p) - \frac{1}{p})\diff y + f'_p(y, p) \diff p\\$
Также как и в предыдущем номере, попытаться вывести зависимость между $p$ и $y$

\subsection{Уравнения высокого порядка, допускающие его понижение}
$F(x, y, ..., y^{(n)}) = 0\\$
Важно: в уравнении порядка m будет m констант. $\\$
1. $y^{(n)} = f(x)$, интегрируем $n$ раз. $\\$
2. $F(x, y^{(k)}, y^{(k+1)}, ..., y^{(n)}) = 0$, ввести замену вида $z = y^{(k)}$, тем самым понижая порядок.$\\$
3. $F(y, y', y'', ..., y^{(n)}) = 0$, ввести замену вида $y' = p(y)$, тем самым получив уравнение с y в качестве переменной. $\\$
В этом случае нужен пересчёт производной: $\\$
$\frac{\diff y}{\diff x} \longrightarrow \frac{\diff p}{\diff y}$: $y'' = \frac{\diff (y')}{\diff x} = \frac{\diff p}{\diff y} \cdot \frac{\diff y}{\diff x} = p' \cdot p: \\ y'' = p'p \\ y''' = p'p^2 + (p')^2p\\$
4. $F(x, ky, ky', ..., ky^{(n)}) = kF(x, y, y', ..., y^{(n)})$, это уравнение, однородное по $y$ и всем его производным.$\\$
Ввести новую функцию через замену вида $y' = yz(x)$. После подстановки исчезнет y как множитель в уравнении. $\\$
5. $F(kx, k^my, k^{m-1}y', ..., k^{m-n}y^{(n)}) = k^{\alpha}F(x, y, ..., y^{(n)})$, обобщённо однородное уравнение. $\\$
Найти m приравнивая все степени слагаемых ($m$ может быть каким угодно, в том числе и 0). $\\$
Ввести замену вида $\begin{cases} x = e^t, \\ y=z(t)e^{mt}\end{cases}\\$
Необходим пересчёт производной.

\subsection{Линейные уравнения с постоянными коэффициентами}
$a_0 y^{(n)} + a_1 y^{(n-1)} + ... + a_n y = F(x)\\$ 
Сначала решить однородное уравнение $a_0 y^{(n)} + a_1 y^{(n-1)} + ... + a_n y = 0\\$
Для этого составить уравнение вида $a_0 \lambda ^n + a_1 \lambda ^{n-1} + ... + a_{n-1} \lambda + a_n = 0$, найти все $\lambda_i$ (корни характеристического многочлена). $\\$
После этого можно выписать решение однородного уравнения: $y_i = Ce^{\lambda_i x}$, если $\lambda_i$ корень кратности 1 или $y_l = (C_l +C_{l+1}x + C_{l+2}x^2 + ... C_{l+m}x^{m-1})e^{\lambda_l x}$, если кратность $\lambda_l$ равна $m$. Если $\lambda_i$ комплексное, то обязательно существует корень $\lambda_j$, комплексно сопряжённый с $\lambda_i$. Это позволяет выписать корни в вещественном виде, воспользовавшись теоремой Эйлера: $y = C_i e^{\alpha x} cos(\beta x) + C_j e^{\alpha x}sin(\beta x) $, если $\lambda_{i,j} = \alpha \pm i \beta $.$\\$
Если кратности $\lambda_{i,j} > 1$, то вместо соотвествующих коэффициентов будут стоять полиномы(как и в вещественном случае с корнем кратности не 1). $\\$
Решением однородного уравнения будет сумма всех полученных решений для каждого $\lambda \\\\$
Частное решение ищется с помощью правила суммы: $ Ly = f + g; \\Ly_1 = f, Ly_2 = g \Rightarrow L(y_1 + y_2) = f + g.\\$
Разбить $F(x)$ на сумму слагаемых $f(x)$, группируя по $e^{\gamma x}\\$
Возможны 2 случая: $\\$
I: Метод неопределённых коэффициентов $\\$
а) $f(x)$ имеет вид $f = P_m(x) \cdot e^{\gamma x} $  ($m$ это кратность), случай вещественного $\gamma$ $\\$
Тогда ответ будет иметь вид $y = x^s Q_m(x) e^{\gamma x}$, где $s$ это кратность $\gamma$ в характеристическом полиноме (0, если $\gamma$ не присутствует среди корней). $\\$
b) $f = e^{\alpha x}(P(x) cos( \beta x) + Q(x) sin(\beta x)), \gamma = \alpha + i \beta $, случай комплексного $\gamma \\$ 
$y = x^s e^{\alpha x} (R_m(x) cos(\beta x) + T_m(x) sin(\beta x)), m = max(deg(P), deg(Q))$, s выбрать по алгоритму из предыдущего пункта. $\\$
Далее представить многочлены в $y$ как $a + bx + cx^2 + ...$, продифференцировать $n$ раз и подставить в исходное уравнение $Ly = f(x)$. Так определятся все коэффициенты многочленов $a, b, ...$, это и будет искомым частным решением. $\\$
II: Метод вариации произвольных постоянных.$\\$
Выписать систему вида 
$\begin{cases}C'_1 y_1 + C'_2 y_2 + ... + C'_n y_n = 0 \\ C'_1 y'_1 + C'_2 y'_2 + ... + C'_n y'_n = 0  \\ a_0(C'_1 y^{(n-1)}_1 + C'_2 y^{(n-1)}_2 + ... + C'_n y^{(n-1)}_n) = f(x) \end{cases} \\$
Решить относительно $C'_i$, потом проинтегрировать каждое $C'_i $. $\\$
$y_i$ это множитель при $C_i$ в выписанном общем решении диффура. $y_i$ это функция от $x$, то есть $C_i$ также должно быть функцией от $x$. $\\$
Ответом будет сумма общего решения и всех частных.

\subsection{Уравнение Эйлера}
$a_0 x^n y^{(n)} + a_1 x^{n-1} y^{(n-1)} + ... + a_n y = F(x) \\$
Очевидно, что левая часть инвариантна относительно $x$. Значит, можно провести замену вида $x = e^t > 0$. Если $f$ не инвариантна относительно $x$, то дополнительно нужно рассмотреть замену $x = -e^t$, иначе просто добавить модуль в конце. $\\$
Здесь необходим пересчёт производной: $\\y' = \frac{\diff y}{\diff t} \cdot \frac{\diff x}{\diff t} = \dot{y} e^{-t} \\ y'' = \frac{\diff (\diff y)}{\diff x^2} = \frac{\diff y'}{\diff t} \cdot e^{-x} = \dot{(y'_t e^{-t})} e^{-t} = e^{-2t}(y''_t - y'_t) \\$
Дальше решать как неоднородное.

\subsection{Линейные системы уравнений (гроб гроб кладбище)}
$\dot{y} = Ay + F(t)\\$
$y = \begin{pmatrix} y_1 \\ y_2 \\ ... \\ y_n \end{pmatrix} \\$
Рассмотреть общее решение уравнения.
Пусть $\lambda$ это корни характеристического многочлена $\operatorname{det}(A - \lambda E) = 0 \\$
Тогда общее решение имеет вид $y_{oo} = \sum_{i = 1}^n C_i e^{\lambda_i t} \nu^i$, где $\nu^i$ это собственные векторы $A$. $\\$
$(A - \lambda_i E) \nu^i = 0$ — уравнение на собственный вектор, если $\lambda_i$ имеет кратность 1. $\\$
Пусть $\lambda_i$ имеет кратность $n_i$. Тогда $m_i = n_i - r_i$ это число л.н.з. собственных векторов для $\lambda_i$, где $r_i = rk(A - \lambda E) $ . $\\$
Тогда:
$\begin{cases}
y_1 = P_{1 m_i}(t) e^{\lambda_i t} \\
... \\
y_n= P_{n_i m_i}(t) e^{\lambda_i t}
\end{cases}$, где все $P_i$ имеют кратность $m_i$. Дальше решить методом неопределённых коэффициентов (выразить полиномы через добавочные буквы, продифференцировать, подставить в исходное уравнение). Итоговые векторы должны зависеть от $n$ разных $C \\$
Чит: $\dot{y} = Ax \Rightarrow y = (E + (A - \lambda E) t + \frac{1}{2}(A - \lambda E)^2 t^2 + ... )C e^{\lambda t}$, где $C = \begin{pmatrix} C_1 \\ ... \\ C_n \end{pmatrix}$, число слагаемых в формуле равно порядку нильпотентности матрицы $A - \lambda E$ (это $m_i$) и $\lambda$ это какой-то корень. $\\\\$
Как и в случае с линейным уравнением, разбить $F(t)$ на сумму $f(t)$, найти частные решения для каждого $f(t)$ и сложить их с общим. $\\$
1) Метод вариации произвольной постоянной. $\\$
Продифференцировать вектор общих решений покомпонентно, учитывая, что $C_i$ это функции.
Подставить в исходную систему $\dot{y} = Ay + f(t)$, заменяя $\dot{y}$ на новый вектор в котором есть $C'$ и $y$ на вектор общих решений. НЕТОЧНО: должны сократиться все чистые $C$, то есть остаться только $C'$ и коэффициенты при них. $\\$
2) Метод неопределённых коэффициентов $\\$
Если $f(t)$ состоит из функций вида $P_m(t) e^{\gamma t}$, то каждое частное решение в векторе будет иметь вид $y_i = e^{\gamma_i t}Q^i_{m+s}(t)$, где $s=0$, если $\gamma_i$ не собственное число, и $s = $ кратность $\gamma_i$ среди $\lambda$ иначе. $\\$Продифференцировать $y$, заменив $Q^i_{m+s}(t)$ на $a^i_0 + a^i_1 t + ... + a^i_{m+s} t^{m+s}$. Подставить в исходное уравнение $\dot{y} = Ay + f(y)$ и определить значения всех коэффициентов в полиномах $Q^i$. $\\$
Если $\gamma$ комплексное, то $f(t)$ должно иметь вид $e^{\alpha t}(P_{m_1} cos(\beta t) + Q_{m_2} sin(\beta t))$ при $\gamma = \alpha \pm i \beta \\$ 
Тогда решение будет иметь вид $y_i = e^{\alpha_i t}(T^i_{max(m_1, m_2) + s}(t) cos(\beta_i t) + R^i_{max(m_1, m_2) + s}(t) sin(\beta_i t))$ в полном соответствии с линейными уравнениями, где $s$ это кратность $\gamma$. $\\$
Точно так же методом неопределённых коэффициентов определяем неизвестные полиномы и выписываем вектор-ответ. $\\\\$
Сумма общего решения и всех частных будет являться ответом на задачу.
\end{document}

